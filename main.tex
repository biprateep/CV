%%%%%%%%%%%%%%%%%%%%%%%%%%%%%%%%%%%%%%%%%
% "ModernCV" CV and Cover Letter
% LaTeX Template
% Version 1.11 (19/6/14)
%
% This template has been downloaded from:
% http://www.LaTeXTemplates.com
%
% Original author:
% Xavier Danaux (xdanaux@gmail.com)
%
% License:
% CC BY-NC-SA 3.0 (http://creativecommons.org/licenses/by-nc-sa/3.0/)
%
% Important note:
% This template requires the moderncv.cls and .sty files to be in the same 
% directory as this .tex file. These files provide the resume style and themes 
% used for structuring the document.
%
%%%%%%%%%%%%%%%%%%%%%%%%%%%%%%%%%%%%%%%%%

%----------------------------------------------------------------------------------------
%	PACKAGES AND OTHER DOCUMENT CONFIGURATIONS
%----------------------------------------------------------------------------------------
\PassOptionsToPackage{colorlinks=true,linkcolor=gray,filecolor=blue, urlcolor=blue,pdfpagemode=FullScreen,pdftitle={Biprateep Dey's CV},}{hyperref}
\documentclass[10pt,a4paper,roman,]{moderncv} % Font sizes: 10, 11, or 12; paper sizes: a4paper, letterpaper, a5paper, legalpaper, executivepaper or landscape; font families: sans or roman

\moderncvstyle{classic} % CV theme - options include: 'casual' (default), 'classic', 'oldstyle' and 'banking'
\moderncvcolor{blue} % CV color - options include: 'blue' (default), 'orange', 'green', 'red', 'purple', 'grey' and 'black'


\usepackage[scale=0.8]{geometry} % Reduce document margins
%\setlength{\hintscolumnwidth}{3cm} % Uncomment to change the width of the dates column
%\setlength{\makecvtitlenamewidth}{10cm} % For the 'classic' style, uncomment to adjust the width of the space allocated to your name

\usepackage{fontawesome5}
\usepackage{academicons}
\usepackage{etaremune}
\usepackage{array}
\usepackage{longtable}
\definecolor{orcidlogocol}{HTML}{A6CE39}

% font loading
% for luatex and xetex, do not use inputenc and fontenc
% see https://tex.stackexchange.com/a/496643
\ifxetexorluatex
\usepackage{fontspec}
\usepackage{unicode-math}
\defaultfontfeatures{Ligatures=TeX}
\setmainfont{Latin Modern Roman}
\setsansfont{Latin Modern Sans}
\setmonofont{Latin Modern Mono}
\setmathfont{Latin Modern Math}

% you may also consider Fira Sans Light for a extra modern look
%\setsansfont[ItalicFont={Fira Sans Light Italic},%
%           BoldFont={Fira Sans},%
%           BoldItalicFont={Fira Sans Italic}]%
%           {Fira Sans Light}%
\else
\usepackage[utf8]{inputenc}
\usepackage[T1]{fontenc}
\usepackage{lmodern}
\fi
%%%%%%USE THIS TO HAVE PAGE NUMBERS %%%%
\usepackage{lastpage}
\rfoot{\addressfont\itshape\textcolor{gray}{Page \thepage\ of \pageref{LastPage}}}
%%%%%%%%%%%%%%%%%%%%%%%%%%%%%%%%%




\urlstyle{same}

%Definition for a custom \mysubsection class
\makeatletter
\NewDocumentCommand{\mysubsection}{sm}{%
  \par\addvspace{1ex}%
  \phantomsection{}% reset the anchor for hyperrefs
  \addcontentsline{toc}{subsection}{#2}%
  {\strut\raggedleft\raisebox{\baseletterheight}{\color{color1}\rule{0.5\hintscolumnwidth}{0.95ex}}\quad}{\strut\subsectionstyle{#2}}%
  \vspace{0.2cm}\par\nobreak\addvspace{.5ex}\@afterheading}% to avoid a pagebreak after the heading
\makeatother
%----------------------------------------------------------------------------------------
%	NAME AND CONTACT INFORMATION SECTION
%----------------------------------------------------------------------------------------

\firstname{Biprateep} % Your first name
\familyname{Dey} % Your last name

% All information in this block is optional, comment out any lines you don't need
\title{Curriculum Vitae}
\address{Department of Statistical Sciences,\\ University of Toronto \\}{Toronto, ON--M5G 1Z5, Canada}
%\mobile{8908691745}
%\phone{(000) 111 1112}
%\fax{(000) 111 1113}
\email{biprateep@pitt.edu}

\homepage{biprateep.github.io}
\ifx\orcidsocialsymbol\undefined
\newcommand*\orcidsocialsymbol {{\aiOrcid~}}
\fi 
\social[orcid][https://orcid.org/0000-0002-5665-7912]{0000-0002-5665-7912}
\social[github]{biprateep} 

%----------------------------------------------------------------------------------------

\begin{document}

\makecvtitle % Print the CV title

%----------------------------------------------------------------------------------------
%	EDUCATION SECTION
%----------------------------------------------------------------------------------------
\section{Employment}
\cventry{2024-present}{Eric and Wendy Schmidt AI in Science Postdoctoral Fellow}{Department of Statistical Sciences}{University of Toronto}{Toronto, Ontario, Canada}{}
\cventry{2024-present}{CITA Postdoctoral Fellow}{Canadian Institute for Theoretical Astrophysics (CITA)}{University of Toronto}{Toronto, Ontario, Canada}{}
\cventry{2024-present}{Dunlap Fellow}{Dunlap Institute for Astronomy and Astrophysics}{University of Toronto}{Toronto, Ontario, Canada}{}
\section{Education}
\cventry{2018-2024}{Ph.D. in Physics}{Department of Physics and Astronomy, University of Pittsburgh}{Pittsburgh, Pennsylvania, USA}{}
{\textbf{Thesis Title:} \textrm{Cosmic cartography: Photometric Redshifts for the Next-Generation of Sky Surveys} \\
\textbf{Advisors:} Prof.~Jeff Newman and Prof.~Brett Andrews}

\cventry{2020}{M.S.~in Physics}{University of Pittsburgh}{Pittsburgh, Pennsylvania, USA}{}{}

\cventry{2018}{Integrated B.Sc.-M.Sc.\ in Physics}{National Institute of Science Education and Research (NISER)}{Bhubaneswar, Odisha, India}{}{\textbf{Thesis Title:} \textrm{Constructing predictors for HI mass in galaxies} \\ \textbf{Advisor:} Prof.\ Nishikanta Khandai}

 % Arguments not required can be left empty



%%%%%%%%%%%%%%%%%%%%%%%%%%%%%%%%%%%%%%%%%%%%%%%%%%%%%%%%%%%%
% RESEARCH EXPERIENCE
%%%%%%%%%%%%%%%%%%%%%%%%%%%%%%%%%%%%%%%%%%%%%%%%%%%%%%%%%%%%

\section{Research Interests}
\cventry{}{Photometric Redshifts of Galaxies; Machine Learning, Statistics, and Uncertainty Quantification for Astrophysics and Cosmology; Galaxy Formation and Evolution}{}{}{}{}


\cventry{\textbf{Publications:}}{\small 6 lead author, 3 significant contributing author, and 49 contributing author \normalfont(list attached)}{}{}{}{}
\cventry{\textbf{Presentations:\ }}{\ \small 22 invited, 20 contributed \normalfont(list attached)}{}{}{}{}



%%%%%%%%%%%%%%%%%%%%%%%%%%%%%%%%%%%%%%%%%%%%%%%%%%%%%%%%%%%%%%%%
%FELLOWSHIPS AND ACHIEVEMETS
%%%%%%%%%%%%%%%%%%%%%%%%%%%%%%%%%%%%%%%%%%%%%%%%%%%%%%%%%%%%%%%%

\section{Awards and Honors}
\cventry{2023}{``Builder'' of the Dark Energy Spectroscopic Instrument (DESI) Collaboration}{}{Builder status is awarded to DESI members in recognition of a long engagement and a significant contribution to the collaboration infrastructure and service work}{}{}
\cventry{2023}{American Physical Society Topical Group on Data Science IMPACT Award for Excellence in Graduate Research}{}{}{}{}
\cventry{2022}{LSST Corporation Enabling Science Fellowship}{}{Funding to attend the 2022 Rubin Observatory's Project and Community Workshop}{}{}
\cventry{2022 - 23}{Andrew Mellon Predoctoral Fellowships}{}{Funding for year long research at the Univ. of Pittsburgh}{}{}
\cventry{2022}{Zaccheus Daniel Predoctoral Fellowship}{}{Funding for summer term research at the Univ. of Pittsburgh}{}{}
\cventry{2021}{PITT-PACC Fellowship}{}{Funding for fall term research at the Univ. of Pittsburgh}{}{}
\cventry{2017}{MITACS Globalink Research Fellowship}{}{Funding for summer research at the Univ. of Alberta}{}{}

\cventry{2014 - 2018}{Kishore Vaigyanik Protsahan Yojana (KVPY) Fellowship}{ Dept. of Science and Tech., Govt. of India}{}{Scholarship for undergraduate studies and research internships}{}
\cventry{2013 - 2014}{INSPIRE Fellowship}{Dept. of Science and Tech., Govt. of India}{}{Scholarship for undergraduate studies}{}
% %%%%%%%%%%%%%%%%%%%%%%%%%%%%%%%%%%%%%%%%%%%%%%%%%%%%%%%%%%%%
% % Grants, telescope and computation time
% %%%%%%%%%%%%%%%%%%%%%%%%%%%%%%%%%%%%%%%%%%%%%%%%%%%%%%%%%%%%
% \section{Telescope/computational grant/Money}
% \cventry{2022}{LSSTC Enabling science grant}{}{}{}{}
% \cventry{2022}{Neocortex}{}{}{}{}
% \cventry{2022}{ACCelerate Money}{}{}{}{}
% \cventry{2022}{HST proposal with David}{}{}{}{}
%%%%%%%%%%%%%%%%%%%%%%%%%%%%%%%%%%%%%%%%%%%%%%%%%%%%%%%%%%%%%%%
%Observing
%%%%%%%%%%%%%%%%%%%%%%%%%%%%%%%%%%%%%%%%%%%%%%%%%%%%%%%%%%%%%%%
% \section{Observing Experience}
% \cvlistitem{10 nights using the Dark Energy Spectroscopic Instrument at Mayall 4-m telescope.}
% \cvlistitem{50 hours with the Giant Metre-wave Radio telescope to search for Pulsars and radio transients}
 

%%%%%%%%%%%%%%%%%%%%%%%%%%%%%%%%%%%%%%%%%%%%%%%%%%%%%%%%%%%%%%%
%Service
%%%%%%%%%%%%%%%%%%%%%%%%%%%%%%%%%%%%%%%%%%%%%%%%%%%%%%%%%%%%%%%

%%%%%%%%%%%%%%%%%%%%

%%%%%%%%%%%%%%%%%%%%

\section{Leadership \& Service}
\cventry{2023}{Member}{}{LSST-DESC Collaboration Council Nominating Committee}{}{}
\cventry{2023 - present}{Research Mentor}{}{Research Mentor for Emma R. Moran, undergraduate student at Univ. of Pittsburgh}{}{} %Name: Emma Moran
\cventry{2023 - present}{Full Member}{}{LSST Dark Energy Science Collaboration}{}{}
\cventry{2023}{Reviewer}{}{International Conference on Machine Learning (ICML) Synergy of Scientific and Machine Learning Modeling}{}{}
\cventry{2023}{Reviewer}{}{Neural Information Processing Systems (NeurIPS) Machine Learning and the Physical Sciences}{}{}
\cventry{2023}{Co-Chair}{}{DESI photo-$z$ Topical Group}{}{}
\cventry{2022}{Reviewer}{}{Neural Information Processing Systems (NeurIPS) Machine Learning and the Physical Sciences}{}{}
\cventry{2022}{Chair}{}{LSST-DESC Collaboration Council Nominating Committee}{}{}
\cventry{2022 - 2023}{Graduate Student Mentor}{}{Mentor for 5 incoming students at the Dept.\ of Physics and Astronomy, University of Pittsburgh}{}{}
\cventry{2022 - 2023}{Member}{}{DESI Committee for Early Career Scientists}{}{}
\cventry{2021 - 2022}{Chairperson}{}{DESI Committee for Early Career Scientists}{}{}
\cventry{2020 - 2022}{Member}{}{DESI Outreach Committee}{}{}
\cventry{2020 - 2021}{Graduate Student Mentor}{}{Mentor for 3 incoming students at the Dept.\ of Physics and Astronomy, University of Pittsburgh}{}{}
\cventry{2020 - 2021}{Coordinator}{}{Astrosnacks, Dept.\ of Physics and Astronomy, University of Pittsburgh}{}{Organized student driven talk and tutorial series}
\cventry{2020 - 2021}{Secretary}{}{Executive Committee, Bengali Association of Pittsburgh}{}{}
\cventry{2019 - 2021}{Coordinator}{}{Astronomy on Tap, Pittsburgh}{}{}

%%%%%%%%%%%%%%%%%%%%%%%%%%%%%%%%%%%%%%
%Telescope and Computing time
%%%%%%%%%%%%%%%%%%%%%%%%%%%%%%%%%%%%%%
\section{Awarded Super-computing Time}
\cventry{2024}{Perlmutter Supercomputer at NERSC}{Photometric Redshifts for LSST and Beyond}{\textbf{1500 CPU Hours and 8000 GPU Hours}}{PI: \textbf{B.~Dey}, Co-I: J.~Newman, B.~Andrews}{}
\cventry{2023}{Perlmutter Supercomputer at NERSC}{Photometric Redshifts for LSST}{\textbf{500 CPU Hours and 2800 GPU Hours}}{PI: \textbf{B.~Dey}, Co-I: J.~Newman, B.~Andrews}{}
\cventry{2022}{Neocortex (Cerebras Wafer Scale Engine) at Pittsburgh Supercomputing Center }{Making the Largest Map of Our Universe}{\textbf{500 machine hours}}{PI: \textbf{B.~Dey}, Co-I: J.~Newman, B.~Andrews, J.~Rajasegaran}{}

\section{Awarded Telescope Time}
\cventry{2023}{Dark Energy Spectroscopic Instrument (DESI)}{Testing ELG Selections for DESI-2}{\textbf{18k fiber hours}}{PI: J. Newman, Co-I: \textbf{B. Dey} and others}{}

\cventry{2023}{Dark Energy Spectroscopic Instrument (DESI)}{Four in one: A consolidated program for DESI-2 and DESI-1b science cases in the COSMOS field}{\textbf{5k fiber hours}}{PI: \textbf{B.~Dey}, Co-PI: A.~Leauthaud, J.~Newman, R.~Wechsler, Y.~Mao}{}

\cventry{2022}{Dark Energy Spectroscopic Instrument (DESI)}{DESI-2 for Deep Spectroscopic Samples for LSST Photo-$z$’s}{\textbf{6k fiber hours}}{PI: \textbf{B.~Dey}, Co-I: J.~Newman, B.~Andrews, R.~Zhou, J.~Myles, J.~McCullough, D.~Gruen, N.~Weaverdyck}{}

\cventry{2022}{Hubble Space Telescope Cycle 30 SNAP Proposal}{Post-starbursts from DESI: Timing quenching and morphological transformation at $1 < z < 1.3$}{\textbf{409 Orbits}}{PI: D. Setton, Co-I: \textbf{B. Dey} and others}{}

\section{Funding}
\cventry{2023}{Nancy Grace Roman Space Telescope Research and Support Participation Opportunities}{Exploiting Deep Learning to Improve Roman Photometric Redshifts}{\textbf{$\sim$\$219k}}{PI: J. Newman, Co-I: \textbf{B. Dey} and others}{}

\cventry{2023}{Nancy Grace Roman Space Telescope Research and Support Participation Opportunities}{A Statistical Framework for Optimizing Roman Spectroscopic Training Sets}{\textbf{$\sim$\$219k}}{PI: J. Newman, Co-I: \textbf{B. Dey} and others}{}

\cventry{2022}{Hubble Space Telescope Cycle 30 SNAP}{Post-starbursts from DESI: Timing quenching and morphological transformation at $1 < z < 1.3$}{\textbf{$\sim$\$203k}}{PI: D. Setton, Co-I: \textbf{B. Dey} and others}{}

\cventry{2021}{2022 ACCelerate Creativity + Innovation Festival}{}{Secured funding (\textbf{$\sim$\$12K}) from the University of Pittsburgh and the Atlantic Coast Conference (ACC) to produce a museum exhibit on \textit{Making the largest Maps of our Universe}}{\textbf{PI: B. Dey}, Co-I: J. Newman}{}

%%%%%%%%%%%%%%%%%%%%%%%%%%%%%%%%%%%%%%%%%%%%%%%%%%%%%%%%%%%%%%%
%Teaching
%%%%%%%%%%%%%%%%%%%%%%%%%%%%%%%%%%%%%%%%%%%%%%%%%%%%%%%%%%%%%%%
% \clearpage
\section{Teaching}
\cventry{April 2024}{Calibration of Uncertainty Estimates for Astronomical Analysis}{}{Presented tutorial at the summer 2024 LSST-ISSC Collaboration meeting}{}{}
\cventry{July 2023}{DESI Spectra Visual Inspection Training}{}{Presented tutorial at the summer 2023 DESI Collaboration meeting}{}{}
\cventry{Summer 2022}{AstroPGH Bootcamp}{}{Presented two lectures on Astropy}{}{}
\cventry{Summer 2021}{AstroPGH-TAMU Bootcamp}{}{Presented two lectures on introductory Numpy}{}{}
\cventry{Summer 2020}{AstroPGH Bootcamp}{}{Presented three lectures on introductory and advanced Numpy}{}{}
\cventry{Spring 2019}{Teaching Assistant}{PHYS 0110: Introduction to Physics 1}{with Prof.~Matteo Broccio and Brian Pardo at Univ. of Pittsburgh}{}{}
\cventry{Fall 2018}{Teaching Assistant}{ASTRON0088: From Stonehenge to Hubble}{with Prof.~Carles Badenes and Prof.~Sandhya Rao at Univ. of Pittsburgh}{}{}

%%%%%%%%%%%%%%%%%%%%%%%%%%%%%%%%%%%%%%%%%%%%%%%%%%%%%%%%%%%%%%
% SOFTWARE
%%%%%%%%%%%%%%%%%%%%%%%%%%%%%%%%%%%%%%%%%%%%%%%%%%%%%%%%%%%%%%
\section{Software}
\begin{center}
    \textit{\small(List of software packages I am the primary developer of)\\}
\end{center}

\cventry{\faGithub}{\texttt{\href{https://github.com/lee-group-cmu/Cal-PIT}{Cal-PIT}}}{\small Python package to produce, diagnose and recallibrate PDFs to ensure conditional coverage}{}{}{}
\cventry{\faGithub}{\texttt{\href{https://github.com/desihub/desigal}{desigal}}}{\small Python package providing standardized utilities to use DESI spectra for studies of galaxies}{}{}{}
\cventry{\faGithub}{\texttt{\href{https://github.com/biprateep/spline_basis}{spline\_basis}}}{\small Python package \texttt{B-spline} and \texttt{I-spline} basis functions to represent PDFs}{}{}{}
%%%%%%%%%%%%%%%%%%%%%%%%%%%%%%%%%%%%%%%%%%%%%%%%%%%%%%%%%%%%%%%
%Outreach
%%%%%%%%%%%%%%%%%%%%%%%%%%%%%%%%%%%%%%%%%%%%%%%%%%%%%%%%%%%%%%%

% \clearpage
\section{Science Communication}

 \begin{etaremune}[leftmargin=40pt,labelsep=10pt]
 \item \textit{Cosmic Cartography: Making the Largest Maps of our Universe}. Allegheny Observatory Public Lecture, Pittsburgh, June 2024.
 \item \textit{Your Guide to the Zodiacs}. Astronomy on Tap, Space Bar, Pittsburgh, February 2024.
 \item \textit{Making the Largest Maps of Our Universe}. Produced an exhibit for the 2022 ACCelerate Creativity + Innovation Festival at the Smithsonian National Museum of American History, April 2022. Secured funding of $\sim$\$10,000. Event attended by more than 10,000 visitors over 3 days. 
 \item \textit{How Stars Helped to Build Human Civilizations}. Biophilia Pittsburgh at the Phipps Conservatory and Botanical Gardens, Pittsburgh, November 2020.
\item \textit{Demystifying Research Internships Abroad: Mitacs Globalink Research Fellowship}. Student Development
Council Talk Series, IISER Bhopal, September 2020.
\item \textit{Mapping the Universe using Sky Surveys}. NISER Astronomy Club Alumni Talk, National Institute of Science Education and Research, August 2020.
\item \textit{Tutor for DESI High: Enabling high school students to use data from DESI} at the 2020 Bay Area Science Festival, 2021 North Carolina Science Festival, 2021 Boston Science Festival, and DESI High@Nepal 2021.
\end{etaremune}

%%%%%%%%%%%%%%%%%%%%%%%%%%%%%%%%%%%%%%%%%%%%%%%%%%%%%%%%%%%%%%%
%Publications
% %%%%%%%%%%%%%%%%%%%%%%%%%%%%%%%%%%%%%%%%%%%%%%%%%%%%%%%%%%%%%%%
\clearpage
\section{List of Publications}
\begin{center}
ADS profile with an up-to-date citation record can be found \href{https://ui.adsabs.harvard.edu/search/p_=0&q=orcid\%3A0000-0002-5665-7912&sort=date\%20desc\%2C\%20bibcode\%20desc}{here.} \\
% \href{https://orcid.org/0000-0002-5665-7912}{\textcolor{orcidlogocol}{\aiOrcid} orcid.org/0000-0002-5665-7912}\\
    \textit{(5 lead author, 4 significant contributing author, and 49 contributing author, 1 in prep)\\}
    
    

\end{center}

\mysubsection{Lead/Significant Contributing Author}
 \begin{etaremune}[leftmargin=40pt,labelsep=10pt]

    \item \textbf{Dey, B.}, Newman, J. A., DESI Collaboration et al., 2024, \textit{in prep}, expected submission by December. 2024,\\
    \textrm{\textit{DESI Deep Spectroscopy for Photometric Redshift Training and Calibration for LSST}}. % \href{https://arxiv.org/abs/2205.14568}{arXiv:2205.14568}.
    
    
    \item  \textbf{Dey, B.}, Zhao, B., Newman, J. A., et al., 2023, Submitted to Annals of Applied Statistics,\\ \textit{Conditionally Calibrated Predictive Distributions by Probability-Probability Map:Application to Galaxy Redshift Estimation and Probabilistic Forecasting}. 
    
    \item Setton, D. J., \textbf{Dey, B.}, Khullar, G., et al., 2023, The Astrophysical Journal, 947, L31, \\ \textit{DESI Survey Validation Spectra Reveal an Increasing Fraction of Recently Quenched Galaxies at $z\sim1$}.
    
    \item Zhou, R., \textbf{Dey, B.}, Newman, J. A., et al., 2023, The Astronomical Journal, 165, 58, \\ \textit{Target Selection and Validation of DESI Luminous Red Galaxies}.
    
    \item  \textbf{Dey, B.}, Newman J. A., Andrews B. H., et al., 2022, Monthly Notices of the Royal Astronomical Society, 515, 5285, \\ \textit{Photometric redshifts from SDSS images with an interpretable deep capsule network}. 

    \item Chen, T. Y., \textbf{Dey, B.*}, Ghosh A., et al., 2022, Proceedings of the US Community Study on the Future of Particle Physics (Snowmass 2021). \\ \textit{Interpretable Uncertainty Quantification in AI for HEP.}
    
    % \href{https://arxiv.org/abs/2208.03284}{arXiv:2208.03284}.
    
   
    \item \textbf{Dey, B.}, Newman, J.A., Andrews, B.H., et al., 2021, Fourth Workshop on Machine Learning and the Physical Sciences (NeurIPS 2021),\\ \textit{Re-calibrating Photometric Redshift Probability Distributions Using Feature-space Regression}. 
    
    % \href{https://arxiv.org/abs/2110.15209}{arXiv:2110.15209}.

    \item \textbf{Dey, B.}, Rosolowsky, E., Cao, Y., et al., 2019, Monthly Notices of the Royal Astronomical Society, 488, 1926, \\ \textit{The EDGE-CALIFA survey: Exploring the star formation law through variable selection}.
    
    % \href{https://arxiv.org/abs/1906.02273}{arXiv:1906.02273}.

   \item Bhattacharjee, S.,\textbf{ Dey, B.}, Mohapatra, A. K., 2018, European Journal of Physics, 39, 035404, \\ \textit{Study of geometric phase using classical coupled oscillators}. 
   
   % \href{https://arxiv.org/abs/2110.15711}{arXiv:2110.15711}.



   
\end{etaremune}
\hspace*{\fill}\textbf{*} Corresponding author
 \mysubsection{Contributing Author}
 \begin{etaremune}[leftmargin=40pt,labelsep=10pt]
 %%%%%%%%%%%%%%%%%%%%%%%%%%%%%%%%%%%%%%%%%%%%%%%%%%%
% Exported using the following ADS string

%     \\item %3G[including \textbf{Dey, B.}], %Y, %J, %V, %p, \\\ \textit{%T}. \n \n
%%%%%%%%%%%%%%%%%%%%%%%%%%%%%%%%%%%%%%%%%%%%%%%%%%%





\item DESI Collaboration, et al.[including \textbf{Dey, B.}], 2024, The Astronomical Journal, 168, 58, \\ \textit{The Early Data Release of the Dark Energy Spectroscopic Instrument}. 
 

\item Zhang, Y., et al.[including \textbf{Dey, B.}], 2024, arXiv e-prints, arXiv:2407.21257, \\ \textit{DESI Massive Post-Starburst Galaxies at $\mathbf{z\sim1.2}$ have compact structures and dense cores}. 
 

\item Wu, X., et al.[including \textbf{Dey, B.}], 2024, arXiv e-prints, arXiv:2407.17809, \\ \textit{Tracing the evolution of the cool gas in CGM and IGM environments through Mg II absorption from redshift z=0.75 to z=1.65 using DESI-Y1 data}. 
 

\item Hadzhiyska, B., et al.[including \textbf{Dey, B.}], 2024, arXiv e-prints, arXiv:2407.07152, \\ \textit{Evidence for large baryonic feedback at low and intermediate redshifts from kinematic Sunyaev-Zel'dovich observations with ACT and DESI photometric galaxies}. 
 

\item Sailer, N., et al.[including \textbf{Dey, B.}], 2024, arXiv e-prints, arXiv:2407.04607, \\ \textit{Cosmological constraints from the cross-correlation of DESI Luminous Red Galaxies with CMB lensing from Planck PR4 and ACT DR6}. 
 

\item Koposov, S. E., et al.[including \textbf{Dey, B.}], 2024, Monthly Notices of the Royal Astronomical Society, \\ \textit{DESI Early Data Release Milky Way Survey Value-Added Catalogue}. 
 

\item Pinon, M., et al.[including \textbf{Dey, B.}], 2024, arXiv e-prints, arXiv:2406.04804, \\ \textit{Mitigation of DESI fiber assignment incompleteness effect on two-point clustering with small angular scale truncated estimators}. 
 

\item White, M., et al.[including \textbf{Dey, B.}], 2024, arXiv e-prints, arXiv:2406.01803, \\ \textit{The clustering of Lyman Alpha Emitting galaxies at z=2-3}. 
 

\item Yantovski-Barth, M. J., et al.[including \textbf{Dey, B.}], 2024, Monthly Notices of the Royal Astronomical Society, 531, 2285, \\ \textit{The CluMPR galaxy cluster-finding algorithm and DESI legacy survey galaxy cluster catalogue}. 
 

\item Khederlarian, A., et al.[including \textbf{Dey, B.}], 2024, Monthly Notices of the Royal Astronomical Society, 531, 1454, \\ \textit{Emission line predictions for mock galaxy catalogues: a new differentiable and empirical mapping from DESI}. 
 

\item Anand, A., et al.[including \textbf{Dey, B.}], 2024, arXiv e-prints, arXiv:2405.19288, \\ \textit{Archetype-Based Redshift Estimation for the Dark Energy Spectroscopic Instrument Survey}. 
 

\item Townsend, A., et al.[including \textbf{Dey, B.}], 2024, arXiv e-prints, arXiv:2405.18589, \\ \textit{Candidate strongly-lensed Type Ia supernovae in the Zwicky Transient Facility archive}. 
 

\item Krolewski, A., et al.[including \textbf{Dey, B.}], 2024, arXiv e-prints, arXiv:2405.17208, \\ \textit{Impact and mitigation of spectroscopic systematics on DESI DR1 clustering measurements}. 
 

\item Yu, J., et al.[including \textbf{Dey, B.}], 2024, arXiv e-prints, arXiv:2405.16657, \\ \textit{ELG Spectroscopic Systematics Analysis of the DESI Data Release 1}. 
 

\item Ross, A. J., et al.[including \textbf{Dey, B.}], 2024, arXiv e-prints, arXiv:2405.16593, \\ \textit{The Construction of Large-scale Structure Catalogs for the Dark Energy Spectroscopic Instrument}. 
 

\item Kong, H., et al.[including \textbf{Dey, B.}], 2024, arXiv e-prints, arXiv:2405.16299, \\ \textit{Forward modeling fluctuations in the DESI LRGs target sample using image simulations}. 
 

\item Karaçaylı, N. G., et al.[including \textbf{Dey, B.}], 2024, arXiv e-prints, arXiv:2405.14988, \\ \textit{CMB lensing and Ly-$\alpha$ forest cross bispectrum from DESI's first-year quasar sample}. 
 

\item Lodha, K., et al.[including \textbf{Dey, B.}], 2024, arXiv e-prints, arXiv:2405.13588, \\ \textit{DESI 2024: Constraints on Physics-Focused Aspects of Dark Energy using DESI DR1 BAO Data}. 
 

\item Calderon, R., et al.[including \textbf{Dey, B.}], 2024, arXiv e-prints, arXiv:2405.04216, \\ \textit{DESI 2024: Reconstructing Dark Energy using Crossing Statistics with DESI DR1 BAO data}. 
 

\item Soumagnac, M. T., et al.[including \textbf{Dey, B.}], 2024, arXiv e-prints, arXiv:2405.03857, \\ \textit{The MOST Hosts Survey: spectroscopic observation of the host galaxies of ~40,000 transients using DESI}. 
 

\item Ramirez-Solano, S., et al.[including \textbf{Dey, B.}], 2024, arXiv e-prints, arXiv:2404.07268, \\ \textit{Full Modeling and Parameter Compression Methods in configuration space for DESI 2024 and beyond}. 
 

\item Ruhlmann-Kleider, V., et al.[including \textbf{Dey, B.}], 2024, arXiv e-prints, arXiv:2404.03569, \\ \textit{High redshift LBGs from deep broadband imaging for future spectroscopic surveys}. 
 

\item Garcia-Quintero, C., et al.[including \textbf{Dey, B.}], 2024, arXiv e-prints, arXiv:2404.03009, \\ \textit{HOD-Dependent Systematics in Emission Line Galaxies for the DESI 2024 BAO analysis}. 
 

\item Mena-Fernández, J., et al.[including \textbf{Dey, B.}], 2024, arXiv e-prints, arXiv:2404.03008, \\ \textit{HOD-Dependent Systematics for Luminous Red Galaxies in the DESI 2024 BAO Analysis}. 
 

\item DESI Collaboration, et al.[including \textbf{Dey, B.}], 2024, arXiv e-prints, arXiv:2404.03002, \\ \textit{DESI 2024 VI: Cosmological Constraints from the Measurements of Baryon Acoustic Oscillations}. 
 

\item DESI Collaboration, et al.[including \textbf{Dey, B.}], 2024, arXiv e-prints, arXiv:2404.03001, \\ \textit{DESI 2024 IV: Baryon Acoustic Oscillations from the Lyman Alpha Forest}. 
 

\item DESI Collaboration, et al.[including \textbf{Dey, B.}], 2024, arXiv e-prints, arXiv:2404.03000, \\ \textit{DESI 2024 III: Baryon Acoustic Oscillations from Galaxies and Quasars}. 
 

\item Brown, Z., et al.[including \textbf{Dey, B.}], 2024, arXiv e-prints, arXiv:2403.18789, \\ \textit{Constraining primordial non-Gaussianity from the large scale structure two-point and three-point correlation functions}. 
 

\item Lamman, C., et al.[including \textbf{Dey, B.}], 2024, Monthly Notices of the Royal Astronomical Society, 528, 6559, \\ \textit{Redshift-dependent RSD bias from intrinsic alignment with DESI Year 1 spectra}. 
 

\item DESI Collaboration, et al.[including \textbf{Dey, B.}], 2024, The Astronomical Journal, 167, 62, \\ \textit{Validation of the Scientific Program for the Dark Energy Spectroscopic Instrument}. 
 

\item Wang, Y., et al.[including \textbf{Dey, B.}], 2023, arXiv e-prints, arXiv:2312.17459, \\ \textit{Measuring the conditional luminosity and stellar mass functions of galaxies by combining the DESI LS DR9, SV3 and Y1 data}. 
 

\item Zhou, R., et al.[including \textbf{Dey, B.}], 2023, Journal of Cosmology and Astroparticle Physics, 2023, 097, \\ \textit{DESI luminous red galaxy samples for cross-correlations}. 
 

\item Moustakas, J., et al.[including \textbf{Dey, B.}], 2023, Astrophysics Source Code Library, ascl:2308.005, \\ \textit{FastSpecFit: Fast spectral synthesis and emission-line fitting of DESI spectra}. 
 

\item Han, J. J., et al.[including \textbf{Dey, B.}], 2023, arXiv e-prints, arXiv:2306.11784, \\ \textit{NANCY: Next-generation All-sky Near-infrared Community surveY}. 
 

\item Prada, F., et al.[including \textbf{Dey, B.}], 2023, arXiv e-prints, arXiv:2306.06315, \\ \textit{The DESI One-Percent Survey: Modelling the clustering and halo occupation of all four DESI tracers with Uchuu}. 
 

\item Hahn, C., et al.[including \textbf{Dey, B.}], 2023, The Astronomical Journal, 165, 253, \\ \textit{The DESI Bright Galaxy Survey: Final Target Selection, Design, and Validation}. 
 


 

\item Guy, J., et al.[including \textbf{Dey, B.}], 2023, The Astronomical Journal, 165, 144, \\ \textit{The Spectroscopic Data Processing Pipeline for the Dark Energy Spectroscopic Instrument}. 
 

\item Raichoor, A., et al.[including \textbf{Dey, B.}], 2023, The Astronomical Journal, 165, 126, \\ \textit{Target Selection and Validation of DESI Emission Line Galaxies}. 
 

\item Alexander, D. M., et al.[including \textbf{Dey, B.}], 2023, The Astronomical Journal, 165, 124, \\ \textit{The DESI Survey Validation: Results from Visual Inspection of the Quasar Survey Spectra}. 
 

\item Chaussidon, E., et al.[including \textbf{Dey, B.}], 2023, The Astrophysical Journal, 944, 107, \\ \textit{Target Selection and Validation of DESI Quasars}. 
 

\item Myers, A. D., et al.[including \textbf{Dey, B.}], 2023, The Astronomical Journal, 165, 50, \\ \textit{The Target-selection Pipeline for the Dark Energy Spectroscopic Instrument}. 
 

\item Myers, A. D., et al.[including \textbf{Dey, B.}], 2023, Astrophysics Source Code Library, ascl:2301.025, \\ \textit{desitarget: Selecting DESI targets from photometric catalogs}. 
 

\item Lan, T.-W., et al.[including \textbf{Dey, B.}], 2023, The Astrophysical Journal, 943, 68, \\ \textit{The DESI Survey Validation: Results from Visual Inspection of Bright Galaxies, Luminous Red Galaxies, and Emission-line Galaxies}. 
 

\item Silber, J. H., et al.[including \textbf{Dey, B.}], 2023, The Astronomical Journal, 165, 9, \\ \textit{The Robotic Multiobject Focal Plane System of the Dark Energy Spectroscopic Instrument (DESI)}. 
 

\item DESI Collaboration, et al.[including \textbf{Dey, B.}], 2022, The Astronomical Journal, 164, 207, \\ \textit{Overview of the Instrumentation for the Dark Energy Spectroscopic Instrument}. 
 
\item Sand, K. R., et al.[including \textbf{Dey, B.}], 2022, The Astrophysical Journal, 932, 98, \\ \textit{Multiband Detection of Repeating FRB 20180916B}. 
 
\item Zhou, R., et al.[including \textbf{Dey, B.}], 2020, Research Notes of the American Astronomical Society, 4, 181, \\ \textit{Preliminary Target Selection for the DESI Luminous Red Galaxy (LRG) Sample}. 
 

\item Sand, K. R., et al.[including \textbf{Dey, B.}], 2020, The Astronomer's Telegram, 13781, 1, \\ \textit{Low-frequency detection of FRB180916 with the uGMRT}. 
 

\item Dutta, S., Khandai, N., Dey, B.[including \textbf{Dey, B.}], 2020, Monthly Notices of the Royal Astronomical Society, 494, 2664, \\ \textit{The population of galaxies that contribute to the H I mass function}. 
\end{etaremune} 

% % \hspace{-0.5in}
% \begin{tabular}{m{2.4cm} m{0.3cm}|m{0.3cm} m{13.8cm}}
%  \emph{Lead/Significant Contributing Author} & & &
 \mysubsection{Lead/Significant Contributing Author}
 \begin{etaremune}[leftmargin=40pt,labelsep=10pt]

    \item \textbf{B. Dey}, J. A. Newman, DESI Collaboration et al. 2023, \textit{in prep}, expected submission by Nov. 2023.\\
    \textrm{\textit{DESI Deep Spectroscopy for Photometric Redshift Training and Calibration for LSST}}. % \href{https://arxiv.org/abs/2205.14568}{arXiv:2205.14568}.
     \\Contribution: Led the development of the methodology and associated code, demonstrated the efficacy of the method on real-world examples, and wrote the paper.
    
    
    \item \textbf{B. Dey}, D. Zhao, J. A. Newman, et al. 2022, Submitted to Annals of Applied Statistics.\\
    \textrm{\textit{Conditionally Calibrated Predictive Distributions by Probability-Probability Map: Application to Galaxy Redshift Estimation and Probabilistic Forecasting}}.  \href{https://arxiv.org/abs/2205.14568}{arXiv:2205.14568}.
     \\Contribution: Led the development of the methodology and associated code, demonstrated the efficacy of the method on real-world examples, and wrote the paper.
    
    \item D. J. Setton, \textbf{B. Dey}, G. Khullar, et al., 2023, ApJL, 947, L31.\\ \textrm{\textit{DESI Survey Validation Spectra Reveal an Increasing Fraction of Recently Quenched Galaxies at $z\sim 1$}}. \href{https://arxiv.org/abs/2212.05070}{arXiv:2212.05070}.
     \\ Contribution: Helped to collect and process the data sets used, worked on the initial prototype of the project, and helped drafting the paper. 
    
    \item R. Zhou, \textbf{B. Dey}, J. A. Newman, et al., 2023, AJ, 165, 58,.\\ \textrm{\textit{Target Selection and Validation of DESI Luminous Red Galaxies}}. \href{https://arxiv.org/abs/2208.08515}{arXiv:2208.08515}.
     \\ Contribution: Worked on optimizing the design of the data set presented in this paper (i.e., optimized the selection cuts, developed ML based stellar mass estimates which were used to ensure mass completeness of the sample and checked dependence of selection on imaging systematics), analyzed telescope data to validate the sample and contributed to the draft of the paper.
    
    \item \textbf{B. Dey}, J. A. Newman, B. H. Andrews, et al., 2022, MNRAS, 515, 4.\\ \textrm{\textit{Photometric redshifts from SDSS images with an interpretable deep capsule network}}. \href{https://arxiv.org/abs/2112.03939}{arXiv:2112.03939}.
     \\ Contribution: Led the development of the methodology and associated code, demonstrated the efficacy of the method on real-world examples, and wrote the paper.
    
    \item T. Chen, \textbf{B. Dey}, A. Ghosh, et al., Proceedings of the US Community Study on the Future of Particle Physics (Snowmass 2021).\\ \textrm{\textit{Interpretable Uncertainty Quantification in AI for HEP.}} \href{https://arxiv.org/abs/2208.03284}{arXiv:2208.03284}.
     \\ Contribution: Drafted the sections pertaining to astrophysics and cosmology and managed the effort to ensure timely submission. 
    
    \item \textbf{B. Dey}, J. A. Newman, B. H. Andrews, et al., 2021, Fourth Workshop on Machine Learning and the Physical Sciences (NeurIPS 2021).\\ \textrm{\textit{Re-calibrating Photometric Redshift Probability Distributions Using Feature-space Regression}}. \href{https://arxiv.org/abs/2110.15209}{arXiv:2110.15209}.
    \\ Contribution: Led the development of the methodology and associated code, demonstrated the efficacy of the method on real-world examples, and wrote the paper.
    
    \item \textbf{B. Dey}, E. Rosolowsky, Y. Cao, et al., 2019, MNRAS, 488, 2.\\ \textrm{\textit{The EDGE-CALIFA survey: exploring the star formation law through variable selection}}. \href{https://arxiv.org/abs/1906.02273}{arXiv:1906.02273}.
    \\ Contribution: Led the data analysis, interpreted the results, and wrote the paper. 
    
    \item S. Bhattacharjee, \textbf{B. Dey} and A.K. Mohapatra, 2018, Eur. J. Phys., 39, 035404.\\ \textrm{\textit{Study of geometric phase using classical coupled oscillators}}. \href{https://arxiv.org/abs/2110.15711}{arXiv:2110.15711}.
    \\ Contribution: Developed components of the experimental apparatus, conducted the experiments, and helped with the writing of the article.
\end{etaremune}
 

% \end{tabular}
 
%  \begin{tabular}{m{2.4cm} m{0.3cm}|m{0.3cm} m{13.8cm}}
 
%  \emph{Contributing Author} & & &
\clearpage
 \mysubsection{Contributing Author}
 \begin{etaremune}[leftmargin=40pt,labelsep=10pt]
 \item R. Zhou, S. Ferraro, M. White et al.[including \textbf{B. Dey}]. Submitted to JCAP.\\ \textrm{\textit{DESI luminous red galaxy samples for cross-correlations}}. \href{https://arxiv.org/abs/2309.06443}{arXiv:2309.06443}
 
 \item M.J. Yantovski-Barth, J.A. Newman, \textbf{B. Dey}, et al. Submitted to MNRAS.\\ \textrm{\textit{The CluMPR Galaxy Cluster-Finding Algorithm and DESI Legacy Survey Galaxy Cluster Catalogue}}. \href{https://arxiv.org/abs/2307.10426}{arXiv:2307.10426}.
 
 \item J. Han, A. Dey, A. Price-Whelan et al.[including \textbf{B. Dey}]. 2023, Submitted to the call for white papers for the Roman Core Community Survey, and to the Bulletin of the AAS. \\ \textrm{\textit{NANCY: Next-generation All-sky Near-infrared Community surveY}}. \href{https://arxiv.org/abs/2306.06315}{arXiv:2306.06315 }.

 \item DESI Collaboration, et al.[including \textbf{B. Dey}]. Submitted to AJ.\\ \textrm{\textit{Validation of the Scientific Program for the Dark Energy Spectroscopic Instrument}}. \href{https://arxiv.org/abs/2306.06307}{arXiv:2306.06307}.
 
  \item DESI Collaboration, et al.[including \textbf{B. Dey}]. Submitted to AJ.\\ \textrm{\textit{The Early Data Release of the Dark Energy Spectroscopic Instrument}}. \href{https://arxiv.org/abs/2306.06308}{arXiv:2306.06308}.

 \item F. Prada, J. Ereza, A. Smith et al.[including \textbf{B. Dey}]. Submitted to MNRAS.
 \\ \textrm{\textit{The DESI One-Percent Survey: Modelling the clustering and halo occupation of all four DESI tracers with Uchuu}}. \href{https://arxiv.org/abs/2306.06308}{arXiv:2306.06308}.
 
 \item E. Chaussidon, C. Y\`eche, N. Palanque-Delabrouille et al.[including \textbf{B. Dey}], 2023, ApJ, 944, 1. \\ \textrm{\textit{Target Selection and Validation of DESI Quasars}}. \href{https://arxiv.org/abs/2208.08511}{arXiv:2208.08511}.
 
 \item C. Hahn, M. J. Wilson, O. Ruiz-Macias et al.[including \textbf{B. Dey}], 2023, AJ, 165, 6. \\ \textrm{\textit{DESI Bright Galaxy Survey: Final Target Selection, Design, and Validation}}. \href{https://arxiv.org/abs/2208.08512}{arXiv:2208.08512}.
 
 \item A. Raichoor, J. Moustakas, J. A. Newman et al.[including \textbf{B. Dey}], 2023, AJ, 165, 3. \\ \textrm{\textit{Target Selection and Validation of DESI Emission Line Galaxies}}. \href{https://arxiv.org/abs/2208.08513}{arXiv:2208.08513}.
 
 \item T. Lan, R. Tojeiro, E. Armengaud et al.[including \textbf{B. Dey}], 2023, ApJ, 943, 1. \\ \textrm{\textit{The DESI Survey Validation: Results from Visual Inspection of Bright Galaxies, Luminous Red Galaxies, and Emission Line Galaxies}}. \href{https://arxiv.org/abs/2208.08516}{arXiv:2208.08516}.
 
 \item D. M. Alexander, T. M. Davis, et al.[including \textbf{B. Dey}], 2023, AJ, 165, 3. \\ \textrm{\textit{The DESI Survey Validation: Results from Visual Inspection of the Quasar Survey Spectra}}. \href{https://arxiv.org/abs/2208.08517}{arXiv:2208.08517}.
 
 \item J. Guy, S. Bailey, A. Kremin, et al.[including \textbf{B. Dey}], 2023, AJ, 165, 4. \\ \textrm{\textit{The Spectroscopic Data Processing Pipeline for the Dark Energy Spectroscopic Instrument}}. \href{https://arxiv.org/abs/2209.14482}{arXiv:2209.14482}.
 
 \item A. D. Myers, J. Moustakas, S. Bailey, et al.[including \textbf{B. Dey}], 2023, AJ, 165, 2. \\ \textrm{\textit{The Target-selection Pipeline for the Dark Energy Spectroscopic Instrument}}. \href{https://arxiv.org/abs/2208.08518}{arXiv:2208.08518}.
 
 \item DESI Collaboration, et al.[including \textbf{B. Dey}], 2022, AJ, 164, 5. \\ \textrm{\textit{Overview of the Instrumentation for the Dark Energy Spectroscopic Instrument}}. \href{https://arxiv.org/abs/2205.10939}{arXiv:2205.10939}.
 
\item K. R. Sand, et al.[including \textbf{B. Dey}], 2022, ApJ, 932, 2. \\ \textrm{\textit{Multiband Detection of Repeating FRB 20180916B}}. \href{https://arxiv.org/abs/2111.02382}{arXiv:2111.02382 }.

\item S Dutta, N Khandai and \textbf{B. Dey}, 2020, MNRAS, 494, 2. \\ \textrm{\textit{The population of galaxies that contribute to the HI mass function}}. \href{https://arxiv.org/abs/1909.03077}{arXiv:1909.03077}.

\item R. Zhou, et al. [including \textbf{B. Dey}], 2020, RNAAS, 4, 10. \\ \textrm{\textit{Preliminary Target Selection for the DESI Luminous Red Galaxy (LRG) Sample}}. \href{https://arxiv.org/abs/2010.11282}{arXiv:2010.11282}.

\item K. R. Sand, et al.[including \textbf{B. Dey}], 2020, ATel, 13781. \\ \textrm{\textit{Low-frequency detection of FRB180916 with the uGMRT}}. \href{https://www.astronomerstelegram.org/?read=13781}{ATel:13781}.
\end{etaremune} 



% \hspace{-0.5in}
% \begin{tabular}{m{2.4cm} m{0.3cm}|m{0.3cm} m{13.8cm}}
%  \emph{Lead/Significant Contributing Author} & & &
 % \mysubsection{Lead/Significant Contributing Author}
 \begin{enumerate}
    \item S. Bhattacharjee, \textbf{B. Dey} and A.K. Mohapatra, 2018, Eur. J. Phys., 39, 035404.\\ \textrm{\textit{Study of geometric phase using classical coupled oscillators}}. \href{https://arxiv.org/abs/2110.15711}{arXiv:2110.15711}.
    \\ Contribution: Developed components of the experimental apparatus, conducted the experiments, and helped with the writing of the article.
    
     \item \textbf{B. Dey}, E. Rosolowsky, Y. Cao, et al., 2019, MNRAS, 488, 2.\\ \textrm{\textit{The EDGE-CALIFA survey: exploring the star formation law through variable selection}}. \href{https://arxiv.org/abs/1906.02273}{arXiv:1906.02273}.
     \\ Contribution: Led the data analysis, interpreted the results, and wrote the paper. 
     
      \item \textbf{B. Dey}, J. A. Newman, B. H. Andrews, et al., 2021, Fourth Workshop on Machine Learning and the Physical Sciences (NeurIPS 2021).\\ \textrm{\textit{Re-calibrating Photometric Redshift Probability Distributions Using Feature-space Regression}}. \href{https://arxiv.org/abs/2110.15209}{arXiv:2110.15209}.
         \\ Contribution: Led the development of the methodology and associated code, demonstrated the efficacy of the method on real-world examples, and wrote the paper.

         
       \item T. Chen, \textbf{B. Dey}, A. Ghosh, et al., Proceedings of the US Community Study on the Future of Particle Physics (Snowmass 2021).\\ \textrm{\textit{Interpretable Uncertainty Quantification in AI for HEP.}} \href{https://arxiv.org/abs/2208.03284}{arXiv:2208.03284}.
         \\ Contribution: Drafted the sections pertaining to astrophysics and cosmology and managed the effort to ensure timely submission. 

         
        \item \textbf{B. Dey}, J. A. Newman, B. H. Andrews, et al., 2022, MNRAS, 515, 4.\\ \textrm{\textit{Photometric redshifts from SDSS images with an interpretable deep capsule network}}. \href{https://arxiv.org/abs/2112.03939}{arXiv:2112.03939}.
         \\ Contribution: Led the development of the methodology and associated code, demonstrated the efficacy of the method on real-world examples, and wrote the paper.
         
         
        \item R. Zhou, \textbf{B. Dey}, J. A. Newman, et al., 2023, AJ, 165, 58.\\ \textrm{\textit{Target Selection and Validation of DESI Luminous Red Galaxies}}. \href{https://arxiv.org/abs/2208.08515}{arXiv:2208.08515}.
         \\ Contribution: Worked on optimizing the design of the data set presented in this paper (i.e., optimized the selection cuts, developed ML based stellar mass estimates which were used to ensure mass completeness of the sample and checked dependence of selection on imaging systematics), analyzed telescope data to validate the sample and contributed to the draft of the paper.

         
        \item D. J. Setton, \textbf{B. Dey}, G. Khullar, et al., 2023, ApJL, 947, L31.\\ \textrm{\textit{DESI Survey Validation Spectra Reveal an Increasing Fraction of Recently Quenched Galaxies at $z\sim 1$}}. \href{https://arxiv.org/abs/2212.05070}{arXiv:2212.05070}.
         \\ Contribution: Helped to collect and process the data sets used, worked on the initial prototype of the project, and helped drafting the paper. 
         

         \item \textbf{B. Dey}, D. Zhao, J. A. Newman, et al. 2022, Submitted to Annals of Applied Statistics.\\
    \textrm{\textit{Conditionally Calibrated Predictive Distributions by Probability-Probability Map: Application to Galaxy Redshift Estimation and Probabilistic Forecasting}}.  \href{https://arxiv.org/abs/2205.14568}{arXiv:2205.14568}.
     \\Contribution: Led the development of the methodology and associated code, demonstrated the efficacy of the method on real-world examples, and wrote the paper.

     
       \item \textbf{B. Dey}, J. A. Newman, DESI Collaboration et al. 2023, \textit{in prep}, expected submission by Nov. 2023.\\
    \textrm{\textit{DESI Deep Spectroscopy for Photometric Redshift Training and Calibration for LSST}}. % \href{https://arxiv.org/abs/2205.14568}{arXiv:2205.14568}.
 \\ Contribution: Conceptualized the project, worked to secure telescope time to acquire the data, led the data analysis and drafted the paper. 
    
   
\end{enumerate}
 


% 

% \clearpage
% \section{List of Relevant Publications with Contributions Described}
% \begin{center}
% ADS profile list of all publications with an up-to-date citation record can be found \href{https://ui.adsabs.harvard.edu/search/p_=0&q=orcid\%3A0000-0002-5665-7912&sort=date\%20desc\%2C\%20bibcode\%20desc}{here.} \\
%     \textit{(Summary: 4 lead author, 4 significant contributing author, and 18 contributing author, 1 in prep)\\}
% \end{center}

% % \hspace{-0.5in}
% \begin{tabular}{m{2.4cm} m{0.3cm}|m{0.3cm} m{13.8cm}}
%  \emph{Lead/Significant Contributing Author} & & &
 % \mysubsection{Lead/Significant Contributing Author}
 \begin{enumerate}
    \item S. Bhattacharjee, \textbf{B. Dey} and A.K. Mohapatra, 2018, Eur. J. Phys., 39, 035404.\\ \textrm{\textit{Study of geometric phase using classical coupled oscillators}}. \href{https://arxiv.org/abs/2110.15711}{arXiv:2110.15711}.
    \\ Contribution: Developed components of the experimental apparatus, conducted the experiments, and helped with the writing of the article.
    
     \item \textbf{B. Dey}, E. Rosolowsky, Y. Cao, et al., 2019, MNRAS, 488, 2.\\ \textrm{\textit{The EDGE-CALIFA survey: exploring the star formation law through variable selection}}. \href{https://arxiv.org/abs/1906.02273}{arXiv:1906.02273}.
     \\ Contribution: Led the data analysis, interpreted the results, and wrote the paper. 
     
      \item \textbf{B. Dey}, J. A. Newman, B. H. Andrews, et al., 2021, Fourth Workshop on Machine Learning and the Physical Sciences (NeurIPS 2021).\\ \textrm{\textit{Re-calibrating Photometric Redshift Probability Distributions Using Feature-space Regression}}. \href{https://arxiv.org/abs/2110.15209}{arXiv:2110.15209}.
         \\ Contribution: Led the development of the methodology and associated code, demonstrated the efficacy of the method on real-world examples, and wrote the paper.

         
       \item T. Chen, \textbf{B. Dey}, A. Ghosh, et al., Proceedings of the US Community Study on the Future of Particle Physics (Snowmass 2021).\\ \textrm{\textit{Interpretable Uncertainty Quantification in AI for HEP.}} \href{https://arxiv.org/abs/2208.03284}{arXiv:2208.03284}.
         \\ Contribution: Drafted the sections pertaining to astrophysics and cosmology and managed the effort to ensure timely submission. 

         
        \item \textbf{B. Dey}, J. A. Newman, B. H. Andrews, et al., 2022, MNRAS, 515, 4.\\ \textrm{\textit{Photometric redshifts from SDSS images with an interpretable deep capsule network}}. \href{https://arxiv.org/abs/2112.03939}{arXiv:2112.03939}.
         \\ Contribution: Led the development of the methodology and associated code, demonstrated the efficacy of the method on real-world examples, and wrote the paper.
         
         
        \item R. Zhou, \textbf{B. Dey}, J. A. Newman, et al., 2023, AJ, 165, 58.\\ \textrm{\textit{Target Selection and Validation of DESI Luminous Red Galaxies}}. \href{https://arxiv.org/abs/2208.08515}{arXiv:2208.08515}.
         \\ Contribution: Worked on optimizing the design of the data set presented in this paper (i.e., optimized the selection cuts, developed ML based stellar mass estimates which were used to ensure mass completeness of the sample and checked dependence of selection on imaging systematics), analyzed telescope data to validate the sample and contributed to the draft of the paper.

         
        \item D. J. Setton, \textbf{B. Dey}, G. Khullar, et al., 2023, ApJL, 947, L31.\\ \textrm{\textit{DESI Survey Validation Spectra Reveal an Increasing Fraction of Recently Quenched Galaxies at $z\sim 1$}}. \href{https://arxiv.org/abs/2212.05070}{arXiv:2212.05070}.
         \\ Contribution: Helped to collect and process the data sets used, worked on the initial prototype of the project, and helped drafting the paper. 
         

         \item \textbf{B. Dey}, D. Zhao, J. A. Newman, et al. 2022, Submitted to Annals of Applied Statistics.\\
    \textrm{\textit{Conditionally Calibrated Predictive Distributions by Probability-Probability Map: Application to Galaxy Redshift Estimation and Probabilistic Forecasting}}.  \href{https://arxiv.org/abs/2205.14568}{arXiv:2205.14568}.
     \\Contribution: Led the development of the methodology and associated code, demonstrated the efficacy of the method on real-world examples, and wrote the paper.

     
       \item \textbf{B. Dey}, J. A. Newman, DESI Collaboration et al. 2023, \textit{in prep}, expected submission by Nov. 2023.\\
    \textrm{\textit{DESI Deep Spectroscopy for Photometric Redshift Training and Calibration for LSST}}. % \href{https://arxiv.org/abs/2205.14568}{arXiv:2205.14568}.
 \\ Contribution: Conceptualized the project, worked to secure telescope time to acquire the data, led the data analysis and drafted the paper. 
    
   
\end{enumerate}
 


% 




%%%%%%%%%%%%%%%%%%%%%%%%%%%%%%%%%%%%%%%%%%%%%%%%%%%%%%%%%%%%%%%
%Talks and Presentations
%%%%%%%%%%%%%%%%%%%%%%%%%%%%%%%%%%%%%%%%%%%%%%%%%%%%%%%%%%%%%%%
\clearpage
\section{List of Presentations}
\begin{center}
    \textit{(20 Invited and 18 Contributed Presentations)\\}
\end{center}
\mysubsection{Invited}

% \begin{longtable}{m{2.4cm} |m{0.3cm} m{15cm}}
%  \emph{Invited} & &
 \begin{etaremune}[leftmargin=40pt,labelsep=10pt]
    \item \textit{Photometric Redshifts for Next-Generation Sky Surveys}. Cosmology X Data Science Meeting, Centre for Computational Astrophysics, Flatiron Institute, USA, November 2023.
    \item \textit{Photometric Redshifts for Next-Generation Sky Surveys}. Yale Cosmology Seminar, Yale University, USA, November 20, 2023.
    \item \textit{Photometric Redshifts for Next-Generation Sky Surveys}. Survey Science Meeting, Princeton University, USA, November 2023.
    \item \textit{Photometric Redshifts for Next-Generation Sky Surveys}. Astrolunch seminar, University of Pittsburgh, USA, October 2023.
    \item \textit{Photometric Redshifts for Next-Generation Sky Surveys}. CCAPP Seminar, The Ohio State University, USA, October 2023.
    \item \textit{Photometric Redshifts for Next-Generation Sky Surveys}. JPL Dark Sector Meeting, NASA Jet propulsion Laboratory, USA, September 2023.
    \item \textit{Photometric Redshifts for Next-Generation Sky Surveys}. Caltech/IPAC Lunch Seminar, Infrared Processing \& Analysis Center (IPAC), Pasadena, USA, September 2023.
    \item \textit{Photometric Redshifts using Interpretable Deep Capsule Networks}. Talk at the DESI@UCL symposium, University College London, London, UK, July 2023.
    \item \textit{Photometric Redshifts using Interpretable Deep Capsule Networks}. Tea Talk, Kavli Institute for Particle Astrophysics and Cosmology, Stanford University, USA, April 2023.
    \item \textit{Calibrated Predictive Distributions for Photometric Redshifts}. Building a physical understanding of galaxy evolution with data-driven astronomy, Kavli Institute for Theoretical Physics, USA, February 2023.
    \item \textit{Calibrated Predictive Distributions}. NSF AI Planning Institute for Data-Driven Discovery in Physics, Carnegie Mellon University, USA, September 2022.
 
    \item \textit{Photometric redshifts for next generation sky surveys}. STAtistical Methods for the Physical Sciences (STAMPS) meeting, Carnegie Mellon University, USA, February 2022.
 
    \item \textit{The Dark Energy Spectroscopic Instrument: One year and 13 million redshifts later}. Plenary talk at Summer 2022 LSST-DESC Collaboration meeting at Kavli Institute for Cosmological Physics, University of Chicago, Chicago, USA, August 2022.
    \item \textit{Photometric redshifts for next-generation sky surveys}. Talk at Astro-Data group meeting, Princeton University, USA, July 2022.
    \item \textit{Photometric redshifts for next-generation sky surveys}. FLASH Lunch talk, University of California, Santa Cruz, USA, June 2022.
    \item \textit{Beyond DESI: Making an even larger map of the Universe}. DESI Lunch, Lawrence Berkeley National Laboratory, USA, June 2022.
    \item \textit{Photometric Redshifts for Next Generation Sky Surveys}. STAtistical Methods for the Physical Sciences (STAMPS) meeting, Carnegie Mellon University, USA, February 2022.
    \item \textit{Photometric Redshifts using Interpretable Deep Capsule Networks}. Institute seminar, Inter-University Centre for Astronomy and Astrophysics (IUCAA), India,  December 2021.
     \item \textit{Capsule Networks: An Astronomer's Perspective}. Break-out session on Deep Learning, Statistical Challenges in Modern Astronomy (SCMA) VII,  June 2021.
    \item \textit{Reducing Photometric Redshift Outliers with Deep Learning}. STAtistical Methods for the Physical Sciences (STAMPS) meeting, Carnegie Mellon University, USA, April 2020.
\end{etaremune}
% \end{longtable}
 
% \vspace{0.5cm}
% \clearpage
\mysubsection{Contributed}
% \begin{tabular}{m{2.4cm} |m{0.3cm} m{15cm}} 
% % \\\multicolumn{2}{c}{} \\
%  \emph{Contributed} & &

 \begin{etaremune}[leftmargin=40pt,labelsep=10pt]
 
%  Photometric Redshifts from SDSS Images with an Interpretable Deep Capsule Network, DESC Photo-z telecon, February 2022
% Re-calibrating Photometric Redshift Probability Distributions Using Feature-space Regression, DESC Photo-z telecon March 2022
% Calibrated Probability Distributions for Photometric Redshifts, DESC Photo-z telecon July 2022
%%%%% TO ADD:
\item \textit{Uncertainty Quantification for the Physical Sciences}. Brown Bag Seminar, Dept. of Statistical Sciences, University of Toronto, Canada, September 2024.
     \item \textit{Calibrated predictive distributions for photometric redshifts}, Talk at STATSTRO 2024: The AIstronomy Revolution, Toronto, Canada, May 2024.
    \item \textit{Calibrated predictive distributions for photometric redshifts}, Talk at the Summer 2024 LSST-ISSC Collaboration Meeting, Boston, USA, April 2024.
    \item \textit{Photometric Redshifts for Next Generation of Sky Surveys}, Thesis Talk at the 243rd Meeting of the American Astronomical Society, New Orleans, USA, January 2023.
    \item \textit{DESI for photo-$z$ Training and Calibration}, Talk at the 2023 Summer DESC Collaboration Meeting. SLAC National Accelerator Laboratory, USA, July 2023.
    \item \textit{Calibrated predictive distributions for photometric redshifts}, Talk at Statistical Challenges in Modern Astronomy (SCMA) VIII, Pennsylvania State University, State College, USA, June 2023.
    \item \textit{DESI Deep Spectroscopy for Photo-z Training and Calibration}. Talk at DESI-2/ Stage-5 Workshop, Napa, USA, March 2023.
    \item \textit{Stellar Masses using Random Forests}. Talk at DESI Collaboration Meeting, Cancun, Mexico, December 2022.
    \item \textit{Calibration of Individual Photometric Redshift Estimates}. Talk at Essential Cosmology for the Next Generation VIII (Cosmology on the Beach), Playa De Carmen, Mexico, November 2022.
    \item \textit{The Dark Energy Spectroscopic Instrument: One year and 13 million redshifts later}. Astrosnacks presentation, University of Pittsburgh, September 2022.
    \item \textit{Calibrated Probability Distributions for Photometric Redshifts}. Poster at Rubin Observatory Project and Community Workshop, Tucson, USA, August 2022.
    \item \textit{Calibrated Probability Distributions for Photometric Redshifts}. Poster and Talk at Summer 2022 LSST-DESC Collaboration meeting at Kavli Institute for Cosmological Physics, University of Chicago, Chicago, USA, August 2022.
    \item \textit{Calibrated Predictive Distributions for Photometric Redshifts}. Poster at ICML 2022 Workshop on Machine Learning for Astrophysics, Baltimore, USA, July, 2022.
    \item \textit{Recalibrating Probability Density Estimates Using Feature-Space Regression}. Refereed talk at the Symposium on Data Science and Statistics, Pittsburgh, USA, June 2022.
    \item \textit{Re-calibrating Photometric Redshift Probability Distributions Using Feature-space Regression}. Poster and Talk at the Fourth Workshop on Machine Learning and the Physical Sciences (NeurIPS 2021), December 2021.
    \item \textit{Interpretable Photometric Redshifts using Deep Capsule Networks}. Talk at the 2nd Symposium on Artificial Intelligence for Science, Industry, and Society (AISIS 2021), October 2021.
     \item \textit{Latent Variable Models: Principal Components}. Talk at AstroPGH-TAMU Bootcamp 2021.
    \item \textit{Interpretable Photometric Redshifts with a DeepCapsule Network}. Poster at Statistical Challenges in Modern Astronomy VII, June 2021.
    \item \textit{Mapping the Universe using Sky Surveys}. Astrosnacks presentation, University of Pittsburgh, July 2020.
    \item \textit{Ancillary Targets: Testing filler samples in Survey validation}. DESI Collaboration meeting, Ohio State University, December 2019.
    \item \textit{LRG \& ELG Imaging systematic Trends} (with A. Raichoor). DESI virtual collaboration meeting, March 2020.
\end{etaremune}
% \\\multicolumn{2}{c}{} \\

% \end{tabular}



%Automatically put last updated
\begin{center}
    Last Updated: \today
\end{center}

\end{document}