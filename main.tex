%%%%%%%%%%%%%%%%%%%%%%%%%%%%%%%%%%%%%%%%%
% "ModernCV" CV and Cover Letter
% LaTeX Template
% Version 1.11 (19/6/14)
%
% This template has been downloaded from:
% http://www.LaTeXTemplates.com
%
% Original author:
% Xavier Danaux (xdanaux@gmail.com)
%
% License:
% CC BY-NC-SA 3.0 (http://creativecommons.org/licenses/by-nc-sa/3.0/)
%
% Important note:
% This template requires the moderncv.cls and .sty files to be in the same 
% directory as this .tex file. These files provide the resume style and themes 
% used for structuring the document.
%
%%%%%%%%%%%%%%%%%%%%%%%%%%%%%%%%%%%%%%%%%

%----------------------------------------------------------------------------------------
%	PACKAGES AND OTHER DOCUMENT CONFIGURATIONS
%----------------------------------------------------------------------------------------

\documentclass[10pt,a4paper,roman]{moderncv} % Font sizes: 10, 11, or 12; paper sizes: a4paper, letterpaper, a5paper, legalpaper, executivepaper or landscape; font families: sans or roman

\moderncvstyle{classic} % CV theme - options include: 'casual' (default), 'classic', 'oldstyle' and 'banking'
\moderncvcolor{blue} % CV color - options include: 'blue' (default), 'orange', 'green', 'red', 'purple', 'grey' and 'black'


\usepackage[scale=0.8]{geometry} % Reduce document margins
%\setlength{\hintscolumnwidth}{3cm} % Uncomment to change the width of the dates column
%\setlength{\makecvtitlenamewidth}{10cm} % For the 'classic' style, uncomment to adjust the width of the space allocated to your name
\usepackage{fontawesome}
\usepackage{etaremune}
\usepackage{array}
\usepackage{lastpage}
\rfoot{\addressfont\itshape\textcolor{gray}{Page \thepage\ of \pageref{LastPage}}}
%----------------------------------------------------------------------------------------
%	NAME AND CONTACT INFORMATION SECTION
%----------------------------------------------------------------------------------------

\firstname{Biprateep} % Your first name
\familyname{Dey} % Your last name

% All information in this block is optional, comment out any lines you don't need
\title{Curriculum Vitae}
\address{Department of Physics and Astronomy,\\ University of Pittsburgh \\}{Pittsburgh, PA--15260, USA}
%\mobile{8908691745}
%\phone{(000) 111 1112}
%\fax{(000) 111 1113}
\email{biprateep@pitt.edu}

\homepage{biprateep.github.io} 

%----------------------------------------------------------------------------------------

\begin{document}

\makecvtitle % Print the CV title

%----------------------------------------------------------------------------------------
%	EDUCATION SECTION
%----------------------------------------------------------------------------------------

\section{Education}
\cventry{2018-present}{PhD. Candidate}{Dept. of Physics and Astronomy, University of Pittsburgh}{Pittsburgh, PA, USA}{}{\textbf{Advisors:} Prof.\ Jeff Newman and Prof.\ Brett Andrews}

\cventry{2020}{M.S.\ in Physics}{University of Pittsburgh}{Pittsburgh, PA, USA}{}{}

\cventry{2018}{Integrated MSc.\ in Physics}{National Institute of Science  Education and Research (NISER)}{Bhubaneswar, Odisha, India}{}{\textbf{Thesis Title:} \textrm{Constructing predictors for HI mass in galaxies} \\ \textbf{Advisor:} Prof.\ Nishikanta Khandai}

 % Arguments not required can be left empty



%%%%%%%%%%%%%%%%%%%%%%%%%%%%%%%%%%%%%%%%%%%%%%%%%%%%%%%%%%%%
% RESEARCH EXPERIENCE
%%%%%%%%%%%%%%%%%%%%%%%%%%%%%%%%%%%%%%%%%%%%%%%%%%%%%%%%%%%%

\section{Research}
\cventry{2019-present}{Photometric redshift estimation using deep learning}{}{with Prof.\ Jeff Newman and Prof.\ Brett Andrews at the University of Pittsburgh, Pittsburgh, USA}{}{\begin{itemize}
\item{Developed a deep neural network based algorithm to incorporate morphological information for galaxy photometric redshifts.}
\item{Developing deep learning based algorithms for well calibrated photometric redshift PDFs.}
\end{itemize}}

\cventry{2019-present}{Target selection and validation for the Dark Energy Spectroscopic Instrument}{}{with Prof.\ Jeff Newman and the DESI Collaboration at the University of Pittsburgh, Pittsburgh, USA}{}{\begin{itemize}
\item{Studied effects of observing systematics on DESI LRG Target selection.}
\item{Developed a low redshift filler sample to improve estimates of cosmological constraints.}
\item{Have worked on other aspects of the survey like developing bad pixel masks for the CCDs and inspection of spectroscopic data quality.}
\end{itemize}}

\cventry{2017-18}{Studying population of galaxies that contribute to the HI mass function}{}{with Prof.\ Nishikanta Khandai at National Institue of Science Education and Research, Bhubaneswar, India}{}{ \begin{itemize}
\item{Studied the relation between properties of galaxies derived from optical and HI mass emission}
\end{itemize}}

\cventry{2017}{Exploring the star formation law using variable selection}{}{with Prof.\ Eric Rosolowsky at the University of Alberta, Edmonton, Canada}{}{ \begin{itemize}
\item{Used machine learning based variable selection methods on data from the EDGE-CALIFA survey to expand the Kennicut-Schmidt law and make it robust and nearly universal.}
\end{itemize}}


\cventry{2016-18}{ Search for Pulsars and Transients using the upgraded GMRT}{}{with Prof. Yashwant Gupta at the National Centre for Radio Astrophysics, TIFR, Pune, India}{}
{\begin{itemize}
\item Worked on developing a GPU based fully real time radio transient detection pipeline for the u\textsc{GMRT}. 
\item Developed software for survey observation planning and management.
\end{itemize}}
%\cventry{}{semester lab}{Guide,at}{ \textbf{title } \\ abstract}{}{}



%\cventry{Time date}{semester lab}{Guide,at}{ \textbf{title } \\ abstract}{}{}


%%%%%%%%%%%%%%%%%%%%%%%%%%%%%%%%%%%%%%%%%%%%%%%%%%%%%%%%%%%%%%%%
%FELLOWSHIPS AND ACHIEVEMETS
%%%%%%%%%%%%%%%%%%%%%%%%%%%%%%%%%%%%%%%%%%%%%%%%%%%%%%%%%%%%%%%%

\section{Awards and Honors}
\cventry{2022}{Andrew Mellon Predoctoral Fellowships}{}{Funding for year long research at the Univ. of Pittsburgh}{}{}
\cventry{2022}{Zaccheus Daniel Predoctoral Fellowship}{}{Funding for Summer term research at the Univ. of Pittsburgh}{}{}
\cventry{2021}{PITT-PACC Fellowship}{}{Funding for fall term research at the Univ. of Pittsburgh}{}{}
\cventry{2017}{MITACS Globalink Research Fellowship}{}{Funding for summer research at the Univ. of Alberta}{}{}

\cventry{2014-2018}{Kishore Vaigyanik Protsahan Yojana (KVPY) Fellowship}{ Dept. of Science and Tech., Govt. of India}{}{Scholarship for undergraduate studies and research internships}{}
\cventry{2013-2014}{INSPIRE Fellowship}{Dept. of Science and Tech., Govt. of India}{}{Scholarship for undergraduate studies}{}
%%%%%%%%%%%%%%%%%%%%%%%%%%%%%%%%%%%%%%%%%%%%%%%%%%%%%%%%%%%%%%%
%Observing
%%%%%%%%%%%%%%%%%%%%%%%%%%%%%%%%%%%%%%%%%%%%%%%%%%%%%%%%%%%%%%%
\section{Observing Experience}
\cvlistitem{10 nights using the Dark Energy Spectroscopic Instrument at Mayall 4-m telescope.}
\cvlistitem{50 hours with the Giant Metre-wave Radio telescope to search for Pulsars and radio transients}
 
%%%%%%%%%%%%%%%%%%%%%%%%%%%%%%%%%%%%%%%%%%%%%%%%%%%%%%%%%%%%%%%
%Teaching
%%%%%%%%%%%%%%%%%%%%%%%%%%%%%%%%%%%%%%%%%%%%%%%%%%%%%%%%%%%%%%%
\section{Teaching}
\cventry{Summer 2021}{AstroPGH-TAMU Bootcamp}{}{Presented two lectures on introductory Numpy}{}{}
\cventry{Summer 2020}{AstroPGH Bootcamp}{}{Presented three lectures on introductory and advanced Numpy}{}{}
\cventry{Spring 2019}{Teaching Assistant}{PHYS 0110: Introduction to Physics 1
}{with Prof.\ Matteo Broccio and Brian Pardo at Univ. of Pittsburgh}{}{}
\cventry{Fall 2018}{Teaching Assistant}{ASTRON0088: From Stonehenge to Hubble}{with Prof.\ Carles Badenes and Prof.\ Sandhya Rao at Univ. of Pittsburgh}{}{}
%%%%%%%%%%%%%%%%%%%%%%%%%%%%%%%%%%%%%%%%%%%%%%%%%%%%%%%%%%%%%%%
%Service
%%%%%%%%%%%%%%%%%%%%%%%%%%%%%%%%%%%%%%%%%%%%%%%%%%%%%%%%%%%%%%%
\section{Leadership \& Service}
\cventry{2021 - present}{Chairperson}{}{DESI Committee for Early Carreer Scientists}{}{}
\cventry{2020 - present}{Member}{}{DESI Outreach Committee}{}{}
\cventry{2020 - 2021}{Graduate Student Mentor}{}{Mentor for 3 incoming students at the Dept.\ of Physics and Astronomy, University of Pittsburgh}{}{}
\cventry{2020 - 2021}{Coordinator}{}{Astrosnacks, Dept.\ of Physics and Astronomy, University of Pittsburgh}{}{}
\cventry{2020 - 2021}{Secretary}{}{Executive Committee, Bengali Association of Pittsburgh}{}{}
\cventry{2019-2021}{Coordinator}{}{Astronomy on Tap-Pittsburgh}{}{}
%%%%%%%%%%%%%%%%%%%%%%%%%%%%%%%%%%%%%%%%%%%%%%%%%%%%%%%%%%%%%%%
%Outreach
%%%%%%%%%%%%%%%%%%%%%%%%%%%%%%%%%%%%%%%%%%%%%%%%%%%%%%%%%%%%%%%
\section{Outreach}
\begin{tabular}{m{2.4cm} m{0.3cm} m{15cm}}
 & & \begin{etaremune}\item \textit{How Stars Helped to Build Human Civilizations}. Biophilia Pittsburgh at the Phipps Conservatory and Botanical Gardens, Pittsburgh, November 2020.
\item \textit{Demystifying Research Internships Abroad: Mitacs Globalink Research Fellowship}. Student Development
Council Talk Series, IISER Bhopal, September 2020.
\item \textit{Mapping the Universe using Sky Surveys}. NISER Astronomy Club Alumni Talk, National Institute of Science Education and Research, August 2020.
\item \textit{Tutor for DESI High: Enabling high school students to use data from DESI} at the 2020 Bay Area Science Festival, 2021 North Carolina Science Festival, 2021 Boston Science Festival, and DESI High@Nepal 2021.
\end{etaremune}
\end{tabular}

%%%%%%%%%%%%%%%%%%%%%%%%%%%%%%%%%%%%%%%%%%%%%%%%%%%%%%%%%%%%%%%
%Talks and Presentations
%%%%%%%%%%%%%%%%%%%%%%%%%%%%%%%%%%%%%%%%%%%%%%%%%%%%%%%%%%%%%%%
\section{Presentations}
\begin{tabular}{m{2.4cm} |m{0.3cm} m{15cm}}
 \emph{Invited} & &  \begin{etaremune}
    \item \textit{Photometric Redshifts for Next Generation Sky Surveys}. STAtistical Methods for the Physical Sciences (STAMPS) meeting, Carnegie Mellon University, USA, February 2022.
    \item \textit{Photometric Redshifts using Interpretable Deep Capsule Networks}. Institute seminar, Inter-University Centre for Astronomy and Astrophysics (IUCAA), India,  December 2021.
     \item \textit{Capsule Networks: An Astronomer's Perspective}. Break-out session on Deep Learning, Statistical Challenges in Modern Astronomy (SCMA) VII,  June 2021.
    \item \textit{Reducing Photometric Redshift Outliers with Deep Learning}. STAtistical Methods for the Physical Sciences (STAMPS) meeting, Carnegie Mellon University, USA, April 2020.
\end{etaremune}
\end{tabular}
 
% \vspace{0.5cm}

\begin{tabular}{m{2.4cm} |m{0.3cm} m{15cm}} 
% \\\multicolumn{2}{c}{} \\
 \emph{Contributed} & &
 \begin{etaremune}
 
%  Photometric Redshifts from SDSS Images with an Interpretable Deep Capsule Network, DESC Photo-z telecon, February 2022
% Re-calibrating Photometric Redshift Probability Distributions Using Feature-space Regression, DESC Photo-z telecon March 2022
%%%%% TO ADD: BErkeley lab, UCSC, Princeton, DESC-v3, ICML4astro, LSST PCW
%%%%ACCELERATE

    \item \textit{Recalibrating Probability Density Estimates Using Feature-Space Regression}. Refereed talk at the Symposium on Data Science and Statistics, Pittsburgh, USA, June 2022.
    
    \item \textit{Re-calibrating Photometric Redshift Probability Distributions Using Feature-space Regression}. Poster and Talk at the Fourth Workshop on Machine Learning and the Physical Sciences (NeurIPS 2021), December 2021.
    \item \textit{Interpretable Photometric Redshifts using Deep Capsule Networks}. Talk at the 2nd Symposium on Artificial Intelligence for Science, Industry, and Society (AISIS 2021), October 2021.
     \item \textit{Latent Variable Models: Principal Components}. Talk at AstroPGH-TAMU Bootcamp 2021.
    \item \textit{Interpretable Photometric Redshifts with a DeepCapsule Network}. Poster at Statistical Challenges in Modern Astronomy VII, June 2021.
    \item \textit{Mapping the Universe using Sky Surveys}. Astrosnacks presentation, University of Pittsburgh, July 2020.
    \item \textit{Ancillary Targets: Testing filler samples in Survey validation}. DESI Collaboration meeting, Ohio State University, December 2019.
    \item \textit{LRG \& ELG Imaging systematic Trends} (with A. Raichoor). DESI virtual collaboration meeting, March 2020.
\end{etaremune}
% \\\multicolumn{2}{c}{} \\

\end{tabular}

%%%%%%%%%%%%%%%%%%%%%%%%%%%%%%%%%%%%%%%%%%%%%%%%%%%%%%%%%%%%%%%
%Publications
%%%%%%%%%%%%%%%%%%%%%%%%%%%%%%%%%%%%%%%%%%%%%%%%%%%%%%%%%%%%%%%

\section{Publications}
\begin{center}
    \textit{(3 lead author, 1 significant contributing author)\\}
\end{center}

\begin{tabular}{m{2.4cm} |m{0.3cm} m{15cm}}
 \emph{Refereed} & &  \begin{etaremune}
    \item \textrm{\textit{Re-calibrating Photometric Redshift Probability Distributions Using Feature-space Regression}}. \textbf{B. Dey}, J. A. Newman, B. H. Andrews, et al., 2021, Fourth Workshop on Machine Learning and the Physical Sciences (NeurIPS 2021).
    \item \textrm{\textit{The population of galaxies that contribute to the HI mass function}}. S Dutta, N Khandai and \textbf{B. Dey}, 2020, MNRAS, 494, 2.
    \item \textrm{\textit{The EDGE-CALIFA survey: exploring the star formation law through variable selection}}. \textbf{B. Dey}, E. Rosolowsky, Y. Cao, et al., 2019, MNRAS, 488, 2.
    \item \textrm{\textit{Study of geometric phase using classical coupled oscillators}}. S. Bhattacharjee, \textbf{B. Dey} and A.K. Mohapatra, 2018, Eur. J. Phys., 39, 035404.
\end{etaremune}
 
 
 
 \\\multicolumn{2}{c}{} \\
 \emph{Non-Refereed} & &
 \begin{etaremune}
    \item \textrm{\textit{Preliminary Target Selection for the DESI Luminous Red Galaxy (LRG) Sample}}. R. Zhou, et al. [26 coauthors including \textbf{B. Dey}], 2020, RNAAS, 4, 10.
    \item \textrm{\textit{Low-frequency detection of FRB180916 with the uGMRT}}. K. R. Sand, et al.[23 coauthors including \textbf{B. Dey}], 2020, ATel, 13781.
\end{etaremune}
 \\\multicolumn{2}{c}{} \\

\end{tabular}

%Automatically put last updated
\begin{center}
    Last Updated: \today
\end{center}

\end{document}