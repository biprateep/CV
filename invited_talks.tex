
% \begin{longtable}{m{2.4cm} |m{0.3cm} m{15cm}}
%  \emph{Invited} & &
 \begin{etaremune}[leftmargin=40pt,labelsep=10pt]
    \item \textit{Photometric Redshifts for Next-Generation Sky Surveys}. TASTY talks, Department of Astronomy and Astrophysics, University of Toronto, Canada, November 2024.
    \item \textit{Photometric Redshifts for Next-Generation Sky Surveys}. Astrophysics Seminar, Université de Montréal, Canada, October 2024.
    \item \textit{Photometric Redshifts for Next-Generation Sky Surveys}. Cosmology Group Meeting, Perimeter Institute, Canada, October 2024.
     \item \textit{DESI Spectroscopy for Photo-$z$ Training and Calibration}. Plenary Talk at Winter 2023 DESI Collaboration Meeting, Hawaii, USA, December 2023.
    \item \textit{Photometric Redshifts for Next-Generation Sky Surveys}. Cosmology X Data Science Meeting, Center for Computational Astrophysics, Flatiron Institute, USA, November 2023.
    \item \textit{Photometric Redshifts for Next-Generation Sky Surveys}. Yale Cosmology Seminar, Yale University, USA, November 2023.
    \item \textit{Photometric Redshifts for Next-Generation Sky Surveys}. Survey Science Meeting, Princeton University, USA, November 2023.
    \item \textit{Photometric Redshifts for Next-Generation Sky Surveys}. Astrolunch seminar, University of Pittsburgh, USA, October 2023.
    \item \textit{Photometric Redshifts for Next-Generation Sky Surveys}. CCAPP Seminar, The Ohio State University, USA, October 2023.
    \item \textit{Photometric Redshifts for Next-Generation Sky Surveys}. JPL Dark Sector Meeting, NASA Jet propulsion Laboratory, USA, September 2023.
    \item \textit{Photometric Redshifts for Next-Generation Sky Surveys}. Caltech/IPAC Lunch Seminar, Infrared Processing \& Analysis Center (IPAC), Pasadena, USA, September 2023.
    \item \textit{The DESI Photometric Redshift Topical Group}, Plenary talk at the 2023 Summer DESI Collaboration Meeting. Durham University, Durham, UK July 2023.
    \item \textit{Photometric Redshifts using Interpretable Deep Capsule Networks}. Talk at the DESI@UCL symposium, University College London, London, UK, July 2023.
    \item \textit{Photometric Redshifts using Interpretable Deep Capsule Networks}. Tea Talk, Kavli Institute for Particle Astrophysics and Cosmology, Stanford University, USA, April 2023.
    \item \textit{Calibrated Predictive Distributions for Photometric Redshifts}. Building a physical understanding of galaxy evolution with data-driven astronomy, Kavli Institute for Theoretical Physics, USA, February 2023.
    \item \textit{Calibrated Predictive Distributions}. NSF AI Planning Institute for Data-Driven Discovery in Physics, Carnegie Mellon University, USA, September 2022.
 
    \item \textit{Photometric redshifts for next generation sky surveys}. STAtistical Methods for the Physical Sciences (STAMPS) meeting, Carnegie Mellon University, USA, February 2022.
 
    \item \textit{The Dark Energy Spectroscopic Instrument: One year and 13 million redshifts later}. Plenary talk at Summer 2022 LSST-DESC Collaboration meeting at Kavli Institute for Cosmological Physics, University of Chicago, Chicago, USA, August 2022.
    \item \textit{Photometric redshifts for next-generation sky surveys}. Talk at Astro-Data group meeting, Princeton University, USA, July 2022.
    \item \textit{Photometric redshifts for next-generation sky surveys}. FLASH Lunch talk, University of California, Santa Cruz, USA, June 2022.
    \item \textit{Beyond DESI: Making an even larger map of the Universe}. DESI Lunch, Lawrence Berkeley National Laboratory, USA, June 2022.
    \item \textit{Photometric Redshifts for Next Generation Sky Surveys}. STAtistical Methods for the Physical Sciences (STAMPS) meeting, Carnegie Mellon University, USA, February 2022.
    \item \textit{Photometric Redshifts using Interpretable Deep Capsule Networks}. Institute seminar, Inter-University Centre for Astronomy and Astrophysics (IUCAA), India,  December 2021.
     \item \textit{Capsule Networks: An Astronomer's Perspective}. Break-out session on Deep Learning, Statistical Challenges in Modern Astronomy (SCMA) VII,  June 2021.
    \item \textit{Reducing Photometric Redshift Outliers with Deep Learning}. STAtistical Methods for the Physical Sciences (STAMPS) meeting, Carnegie Mellon University, USA, April 2020.
\end{etaremune}
% \end{longtable}